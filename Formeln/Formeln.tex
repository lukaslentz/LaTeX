%
\documentclass[a4paper,12pt]{scrartcl}
%
\usepackage[ngerman]{babel}%Als Sprache wird deutch ausgew�hlt
\usepackage[latin1]{inputenc}% westliche 256-Bit-Codierung f�r Schriften, erm�glicht die Verwendung von Umlauten
\usepackage[T1]{fontenc}%Darstellung von Umlauten als ein Zeichen
\usepackage{lmodern}%Verwendung der latin modern schriftfamilie
\usepackage{geometry}%M�glichkeit Geomertry der Seite einzustelleb
\geometry{top=20mm, left=25mm, bottom=20mm, right=20mm, includeheadfoot}
\usepackage{amsmath}%Zus�tzliche Befehle zur Darstrellung von Formeln
%
\begin{document}
%
%
%******************************************************************
\section{Formeln in \LaTeX}
%******************************************************************
%
\subsection{Zus�tzliche Packete}
%
Auch ohne zus�tliche Packete k�nnen Formeln verwendet werden. Durch das Einbinden von \texttt{amsmath} erweitern sich die M�glichkeiten allerdings sehr, weswegen dieses Packet fast immer geladen wird.
%
\subsection{Formeln im Text}
%
Sollen innerhalb des Textes kurze Formeln dargetellt werden, kann dies erreicht werden, indem  eine Mathematikumgebung erzeugt wird, welche nicht abger�ckt ist. 
Beginn und Ende einer solchen Umgebung werden durch ein Dollarzeichen \texttt{\$} gekennzeichnet. 
Das Ergebnis sieht dann z.B. so  $\cos(45)=\frac{1}{2}\sqrt{2}$ aus.
%
\subsection{Abger�ckte Formeln}
%
Um abger�ckte Formeln zu erstellen stehen mehrere Umgebungen zur Verf�gung, deren Beginn und Ende durch die Schl�sselw�rter \texttt{\textbackslash begin\{...\}} und \texttt{\textbackslash end\{...\}} gekennzeichnet werden.
Die wohl am h�ufigsten verwendete Umgebung ist die \texttt{equation}-Umgebung.
Die Eingabe von
%
\\
\\
\texttt{\textbackslash begin\{equation\}
\\\textbackslash cos(45)=\textbackslash frac\{1\}\{2\}\textbackslash sqrt\{2\}
\\\textbackslash label\{formeleins\}
\\\textbackslash end\{equation\}}
%
\\
\\
%
liefert das Ergebnis 
%
\begin{equation}
\cos(45)=\frac{1}{2}\sqrt{2}
\label{formeleins}
\end{equation}
%
wobei mit \texttt{\textbackslash label\{formeleins\}} der Formel ein Name gegeben wird, der verwendet werden kann, um diese Formel durch den Befehl \texttt{\textbackslash eqref\{formeleins\}} zu Referenzieren:
In Formel \eqref{formeleins} wird ein einfacher trigonometrischer Zusammenhang dargestellt.
\\
Sollen mehrere zusammenh�ngende Formeln dargestellt werden, kann dies durch
%
\\
\\
\texttt{\textbackslash begin\{align\}
\\\textbackslash cos(0)\&=\textbackslash frac\{1\}\{2\}\textbackslash sqrt\{4\}\textbackslash label\{eq:cos0\}\textbackslash\textbackslash
\\\textbackslash cos(30)\&=\textbackslash frac\{1\}\{2\}\textbackslash sqrt\{3\}\textbackslash label\{eq:cos30\}\textbackslash\textbackslash
\\\textbackslash cos(45)\&=\textbackslash frac\{1\}\{2\}\textbackslash sqrt\{2\}\textbackslash label\{eq:cos45\}\textbackslash\textbackslash
\\\textbackslash cos(60)\&=\textbackslash frac\{1\}\{2\}\textbackslash sqrt\{1\}\textbackslash label\{eq:cos60\}\textbackslash\textbackslash
\\\textbackslash cos(60)\&=\textbackslash frac\{1\}\{2\}\textbackslash sqrt\{0\}\textbackslash label\{eq:cos90\}
\\\textbackslash end\{align\}}
%
\\
\\
oder
\\
\\
\texttt{\textbackslash begin\{subequations\}
\\\texttt{\textbackslash begin\{align\}
\\\textbackslash cos(0)\&=\textbackslash frac\{1\}\{2\}\textbackslash sqrt\{4\}\textbackslash label\{seq:cos0\}\textbackslash\textbackslash
\\\textbackslash cos(30)\&=\textbackslash frac\{1\}\{2\}\textbackslash sqrt\{3\}\textbackslash label\{seq:cos30\}\textbackslash\textbackslash
\\\textbackslash cos(45)\&=\textbackslash frac\{1\}\{2\}\textbackslash sqrt\{2\}\textbackslash label\{seq:cos45\}\textbackslash\textbackslash
\\\textbackslash cos(60)\&=\textbackslash frac\{1\}\{2\}\textbackslash sqrt\{1\}\textbackslash label\{seq:cos60\}\textbackslash\textbackslash
\\\textbackslash cos(60)\&=\textbackslash frac\{1\}\{2\}\textbackslash sqrt\{0\}\textbackslash label\{seq:cos90\}
\\\textbackslash end\{align\}}
\\\textbackslash label\{seq:cos\}
\\\textbackslash end\{subequations\}}
%
\\
\\
erreicht werden.
Die Ausgabe von Variante 1 ist
%
\begin{align}
\cos(0)&=\frac{1}{2}\sqrt{4}\label{eq:cos0}\\
\cos(30)&=\frac{1}{2}\sqrt{3}\label{eq:cos30}\\
\cos(45)&=\frac{1}{2}\sqrt{2}\label{eq:cos45}\\
\cos(60)&=\frac{1}{2}\sqrt{1}\label{eq:cos60}\\
\cos(90)&=\frac{1}{2}\sqrt{0}\label{eq:cos90}
\end{align}
%
und von Variante 2
%
\begin{subequations}
\begin{align}
\cos(0)&=\frac{1}{2}\sqrt{4}\label{seq:cos0}\\
\cos(30)&=\frac{1}{2}\sqrt{3}\label{seq:cos30}\\
\cos(45)&=\frac{1}{2}\sqrt{2}\label{seq:cos45}\\
\cos(60)&=\frac{1}{2}\sqrt{1}\label{seq:cos60}\\
\cos(90)&=\frac{1}{2}\sqrt{0}\label{seq:cos90}
\end{align}
\label{seq:cos}
\end{subequations}
%
der Unterschied besteht also darin, wie die Gleichungen nummeriert werden, was sich auch auf die Referenzierung auswirkt: Gleichungen \eqref{eq:cos30} und \eqref{seq:cos30} treffen die selbe Aussage.  
Mit dem Zeichen \texttt{\&} wird festgelegt, welche Stellen der Gleichung zueinander ausgerichtet werden sollen. 
%
\end{document}
%
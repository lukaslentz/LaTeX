%
\documentclass[12pt,a4paper]{scrartcl}
%
\begin{document}
%
\section{Einleitung}
Das hier ist die einfachste Version von einem Dokument. 
Wenn Sie bislang noch nicht mit \LaTeX\, gearbeitet haben, sind Sie hier genau richtig.
%
\section{Voraussetzungen}
Wenn Sie \LaTeX\, erstmal ausprobieren möchten brauchen Sie zunächst keine Installationen vorzunehmen und einfach mit overleaf (https://de.overleaf.com/) erste Versuche unternehmen.
Falls Sie dann richtig einsteigen wollen, müssen Sie dann aber zwei Dinge auf Ihrem Rechner installieren.
%
\subsection{TeX-Distribution}
Damit Sie mit \LaTeX\, arbeiten können, brauchen Sie eine TeX-Distribution.
Diese beinhaltet einen Compiler und zahlreiche Bibliotheken, die je nach Bedarf eingebunden werden können.
Üblicherweise werden je nach Betriebssystem folgende Distributionen verwendet:
%
\begin{itemize}
\item Windows: MiKTeX (https://miktex.org/)
\item Linux: TeX Live (https://tug.org/texlive/)
\end{itemize}
%
\subsection{Editor}
Zusätzlich zu der Installation der Distribution empfielt sich die Installation eines Editors, welcher das Arbeiten mit \LaTeX\, komfortabler gestaltet. 
Hier fällt die Entscheidung schon nicht mehr so leicht, da zahlreiche Möglichkeiten existieren und je nach Geschmack der eine oder der andere als besser empfunden wird.
Das Gute ist, dass keiner der prominenten Editoren (Internet-Suche) wirklich schlecht ist und es immer möglich ist zu wechseln.
Ich selbst benutze akteull Texmaker (https://www.xm1math.net/texmaker/).
%
\end{document}
%
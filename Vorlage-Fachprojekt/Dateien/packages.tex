%
\usepackage[ngerman]{babel} % Deutsche Bezeichnungen bei Inhaltsangabe etc
\usepackage[T1]{fontenc}    % andere Schriftsatzkodierung f�r richtige Silbentrennung bei Umlauten
%
\usepackage[]{geometry}%Seitengeometrie
\geometry{ 
					 top         = 20 mm,     % Rand unten
					 bottom      = 20 mm,     % Rand unten
					 left        = 20 mm,     % Rand links
					 right       = 20 mm,     % Rand rechts
           footskip    = 10 mm,     % Abstand Unterkante Text bis Unterkante Fu�zeile
           headsep     =  0 mm      % Abstand Unterkante Kopfzeile bis Oberkante Text
					}
%
\setlength\parindent{0pt}
%
\usepackage[onehalfspacing]{setspace}%Zeilenabstand
%
\usepackage{graphicx} % zum Einbinden von Graphiken
%
\usepackage{amsmath,amsthm,amssymb} % Mathematik Umgebung 
\usepackage{icomma} % Intelligentes Komma, das den richtigen Abstand zwischen Dezimalzahlen als auch in Formeln w�hlt.
%
\usepackage[locale = DE,space-before-unit=true,per-mode = symbol]{siunitx} % Bessere Einheiten
%%
\usepackage{xcolor}
\definecolor{ucbblue}{RGB}{6,13,141}
\definecolor{ucbgreen}{RGB}{67,176,42}
%
\usepackage[breaklinks=true,colorlinks=true,linkcolor=ucbblue,urlcolor=ucbblue,citecolor=red]{hyperref} % Verlinkungen
%
\usepackage[]{natbib} % Zitieren und Literraturverzeichnis
%
\usepackage[most]{tcolorbox} % Erstellen farbiger Boxen 
%
\usepackage{fancyhdr} % Fu� und Kopfzeilen anpassen
%
\usepackage{lastpage} % Nummer der Seiten
%
\usepackage{tikz}
\usetikzlibrary{positioning}
\usetikzlibrary{shapes.geometric, arrows}
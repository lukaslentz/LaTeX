%
\documentclass[12pt,a4paper]{scrartcl}
%
\begin{document}
%
%******************************************************************
\section{Einleitung}
%******************************************************************
%
Wenn Sie bislang noch nicht mit \LaTeX\, gearbeitet haben, finden Sie hier die wichtigsten Informationen, die Sie zum Einstieg benötigen.
Die weiterführenden Themen werden dann in gesonderten Dokumenten behandelt. 
%
%******************************************************************
\section{Voraussetzungen}
%******************************************************************
%
Wenn Sie \LaTeX\, erstmal ausprobieren möchten, brauchen Sie zunächst keine Installationen vorzunehmen und können einfach mit overleaf (https://de.overleaf.com/) erste Versuche unternehmen.
Falls Sie dann richtig einsteigen wollen, müssen Sie zwei Dinge auf Ihrem Rechner installieren.
%
\subsection{TeX-Distribution}
Damit Sie mit \LaTeX\, arbeiten können, brauchen Sie eine TeX-Distribution.
Diese beinhaltet einen Compiler und zahlreiche Bibliotheken, die je nach Bedarf eingebunden werden können.
Üblicherweise werden je nach Betriebssystem folgende Distributionen verwendet:
%
\begin{itemize}
\item Windows: MiKTeX (https://miktex.org/)
\item Linux: TeX Live (https://tug.org/texlive/)
\end{itemize}
%
\subsection{Editor}
Zusätzlich zu der Installation der Distribution empfielt sich die Installation eines Editors, welcher das Arbeiten mit \LaTeX\, komfortabler gestaltet. 
Hier fällt die Entscheidung schon nicht mehr so leicht, da zahlreiche Optionen zur Verfügung stehen und je nach Geschmack die eine oder die andere als besser empfunden wird.
Das Gute ist, dass keiner der prominenten Editoren (Internet-Suche) wirklich schlecht ist und immer die Möglichkeit besteht zu wechseln.
Ich selbst benutze momentan Texmaker (https://www.xm1math.net/texmaker/), habe aber nicht viele getestet.
%
%******************************************************************
\section{Erste Schritte}
%******************************************************************
%
Sobald Sie die Distribution und Editor installiert haben, können Sie ihre ersten Dokumente erstellen. 
Ein Dokument startet immer mit der Angabe, was für eine Art von Dokument erstellt werden soll. 
Die Auswahl erfolgt über den Befehl
%
\begin{verbatim}
\documentclass[12pt,a4paper]{scrartcl}
\end{verbatim}
%
der nun genauer erklärt werden soll.
%
%
\end{document}
%